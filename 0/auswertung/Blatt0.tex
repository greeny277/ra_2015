\documentclass{article}
% -------- Umlaute korrekt ----------------
\usepackage[latin1, utf8]{inputenc}
\usepackage[ngerman]{babel}
%-------------------------------------------

% Einrueckung unterbinden nach Absatz
\setlength{\parindent}{0pt}

\DeclareMathSizes{10}{10}{10}{10}
\title{RA -- R\"U Blatt 0}
\author{Christian Bay, QUED}
\date{\today}
\begin{document}
\maketitle
\section*{Aufgabe 2}

\textbf{Aufrufparameter:} ./Jacobi 6000 500

\begin{center}
	\begin{tabular}{| c || c | c | c |}
		LUPS & O0 & O2 & O3 \\
		\hline \hline
		gcc  & 21-26  & 103-105  & 180-190 \\
		\hline
		icc  & 22-27  & 125-132  & 119-133 \\
		\hline
	\end{tabular}
\end{center}

\subsection*{GCC}
\begin{enumerate}
	\item{ \textbf{gcc -- O0}:
			\begin{itemize}
				\item Labels verteilt $\rightarrow$ viele Sprungoperationen
				\item Kleiner Init Teil: Parameter auf Stack in Register kopiert
				\item Ergebnisse aller arithmetischen Operationen werden zwischengespeichert
			\end{itemize}
		}
	\item{ \textbf{gcc -- O2}:
			\begin{itemize}
				\item Um $\approx 10\%$ k\"urzer als O0 Variante
				\item Gr\"osserer Init Part: Register werden in Abh"angigkeit
					von anderen Registerinhalten gesetzt $\rightarrow$ schnellerer Arrayzugriff
				\item Ergebnisse der arithmetischen Operationen werden so wenig
					zwischengespeichert wie m\"glich $\rightarrow$ CISC Befehle
				\item Labels sind dem Programmverlauf angeordnet:\\
					innere Schleifenvariable erh"ohen $\rightarrow$ innerer Schleifen-Body $\rightarrow$
					\"aussere Schleifenvariable erh\"ohen\\
					$\longrightarrow$ Weniger Spr\"unge
			\end{itemize}
		}
	\item{ \textbf{gcc -- O3}:
			\begin{itemize}
				\item Assemblercode ist um das f\"unfache L"anger im Vergleich
					zu O0 und O2.
				\item \# Labels: 86
				\item In einem Zyklus werden viel mehr Werte aus Grid geladen\\
					$\rightarrow$ Kein doppeltes Laden von Werten
			\end{itemize}
		}
\end{enumerate}
\subsection*{ICC}
Allgemein: Bei \textit{jump} Befehlen, steht als Kommentar die Wahrscheinlichkeit,
des Sprungs.
\begin{enumerate}
	\item{ \textbf{icc - O0}:
			\begin{itemize}
				\item Normaler Assembler Code. Zwischenergebnisse werden in Registern
					gespeichert. Keine gr\"osseren Optimierungen.
			\end{itemize}
		}
	\item{ \textbf{icc - O2}:
		}
	\item{ \textbf{icc - O3}:
		}
\end{enumerate}
\end{document}
